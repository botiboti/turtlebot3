\documentclass{article}
\usepackage[utf8]{inputenc}
\usepackage{hyperref}
\bibliographystyle{plain}

\title{Turtlebot3 docs}
\author{}
\date{}

\begin{document}
\maketitle

% ez a hasznos infó rész még csak piszkozat, törlöm majd miután kimerítettem az egyes témákat
\section{Hasznos információk}
    ROS bevezető:
    \begin{enumerate}
        \item \href{http://wiki.ros.org/ROS/Tutorials}{ROS}
    \end{enumerate}
    
    TurtleBot3 - a robot összerakásához az útbaigazítás itt található meg, valamint a kezelése a ROS környezetben billentyűzettel, példaprogramokkal, joystick-el stb.:
    \begin{enumerate}
        \item \href{https://emanual.robotis.com/docs/en/platform/turtlebot3/overview/}{TurtleBot3 Overview}
    \end{enumerate}
    
    Rospy - Python kliens a ROS-hoz. Az első linken példaprogramok találhatóak, a második linken a dokumentáció, hogyan kell feliratkozni egy ROS topic-ra, hogyan kell egy topic-ot létrehozni, stb. Itt példázva látható hogyan lehet egy egyszerű script-tel irányítani a robotot:
    \begin{enumerate}
        \item \href{https://github.com/markwsilliman/turtlebot/} {A "getting started" guide for developers interested in robotics}
        \item \href{http://wiki.ros.org/rospy/Overview}{Rospy doksi}
    \end{enumerate}
    
    OpenCR: - A motrokat irányító modul, a motorkontroller. Itt fontos megemlíteni, hogy több könyvtár is van mellyel a motrokat irányíthatjuk. A motrokat tudjuk konfigurálni különböző vezérlési módra, állíthatunk baud rate-t, stb.:
    \begin{enumerate}
        \item \href{https://www.youtube.com/watch?v=0_M0Da9SHDw}{Dynamixel2Arduino - gyors és egyszerű vezérlése a motroknak Arduino IDE-ben}
        \item \href{https://emanual.robotis.com/docs/en/software/dynamixel/dynamixel_sdk/overview/}{Dynamixel sdk - ezt a könyvtárat használja a ROS}
    \end{enumerate}
    
    Dynamixel motrok:
    \begin{enumerate}
        \item \href{https://emanual.robotis.com/docs/en/dxl/x/xl430-w250/}{Motorleírás}
    \end{enumerate}

\newpage

\section{ROS}
\subsection{Ismertető}
A ROS-ban a számítások elvégzéséhez ROS-csomópontok néven ismert szoftverkomponenseket használják. A ROS-csomópontok, a mester, a paraméterkiszolgáló, az üzenetek, a témák, a szolgáltatások és a táskák a számítási gráf központi elemei.
A gráf minden egyes fogalma más-más módon járul hozzá a gráfhoz.

A \href{http://wiki.ros.org/ros_comm}{ros\_comm} nevű verem tartalmazza a ROS kommunikációval kapcsolatos csomagjait, beleértve az olyan alapvető klienskönyvtárakat, mint a roscpp és a rospython, valamint az olyan fogalmak megvalósítását, mint a témák, csomópontok, paraméterek és szolgáltatások. Ezek a fogalmak további vizsgálatához ez a verem olyan eszközöket is tartalmaz, mint a rostopic, rosparam, rosservice és rosnode.

A ROS kommunikációs middleware csomagok együttesen alkotják a ROS Graph réteget, a \href{http://wiki.ros.org/ros_comm}{ros\_comm} stack tartalmazza ezeket a csomagokat.

Az ROS gráfok fogalmai:
\begin{itemize}
    \item Csomópontok: A számításokat végző folyamatokat csomópontoknak nevezzük, melyek létrehozásához a roscpp és rospy ROS klienskönyvtárakat használjuk. Ezekben a kommunikáció több formáját is megvalósíthatjuk a klienskönyvtári API-k segítségével. Egy robotban számos csomópont lesz a különböző feladatok elvégzésére, valamint ROS kommunikációs protokollok segítségével kommunikálhatnak egymással és oszthatnak meg adatokat. A ROS-csomópontok egyik célja, hogy inkább kis folyamatokat építsünk ki minimális funkcionalitással, mint komplexet, emiatt egyszerűen hibakereshetők.
    \item Mester: A ROS Master(mester) névkeresést és regisztrációt kínál a többi csomópont számára, enélkül a csomópontok nem tudják megtalálni egymást, nem tudnak kommunikálni egymással, és nem tudnak szolgáltatásokat használni. Egy elosztott rendszerben a mestert egyetlen számítógépen kell futtatni, majd a távoli csomópontok ehhez a mesterhez tudnak csatlakozni, hogy megtalálják egymást.
    \item  Paraméter-kiszolgáló: A paraméterkiszolgáló használatával egy helyen tarthatja az adatokat, ennek segítségével minden csomópont hozzáférhet ezekhez az értékekhez és ellenőrizheti azokat. A ROS Master tartalmaz egy paraméterkiszolgálót.
    \item Üzenetek: Az üzenetek a csomópontok közötti kommunikáció elsődleges eszközei. Egyszerűen fogalmazva, az üzenet egy olyan adatszerkezet, amely egy tipizált mezővel rendelkezik, amely egy adatgyűjteményt tartalmazhat, és továbbítható egy másik csomópontnak. A ROS-üzenetek egész, lebegőpontos, Boolean és más általános primitív típusokat támogatnak, de szerkeszthetünk saját üzenettípust is akár.
    \item Témák: A ROS-ban minden kommunikáció egyedi névvel rendelkező topic-okon (témakon), megnevezett buszokon keresztül történik. Azt mondhatjuk, hogy egy csomópont egy topicot publikál, amikor egy topicon keresztül üzenetet küld, valamint feliratkozik egy topicra, amelyen keresztül üzeneteket kap. A feliratkozó csomópont és a publikáló csomópont nem tudnak egymásról, így olyan témákra is lehet feliratkozni, amelyeknek nincs kiadója. Más szóval az információ létrehozása és fogyasztása különálló folyamat. Amíg egy csomópont rendelkezik a megfelelő üzenettípussal, addig hozzáférhet a témához és küldhet rajta keresztül adatokat.
    \item Szolgáltatások: Ha egy robotalkalmazás kérés-válasz interakciót igényel, a publikálás/feliratkozás megközelítés önmagában nem biztos, hogy elegendő. A kérés/válasz típusú interakcióra akkor lehet szükség, ha elosztott rendszerrel dolgozunk, mivel a publikálás/feliratkozás architektúra lényegében egyirányú szállítási rendszer. Ezekben a helyzetekben ROS-szolgáltatásokat használnak. Lehetséges egy olyan szolgáltatásdefiníció, amely két részből áll, egy a kérésekhez és egy a válaszokhoz. A ROS szolgáltatások segítségével létrehozhatunk egy kiszolgáló csomópontot és egy kliens csomópontot. Amikor az ügyfélcsomópont kérő üzenetet küld a kiszolgálónak, az válaszol, és az eredményt átadja az ügyfélnek. A kiszolgáló csomópont egy név alatt nyújtja a szolgáltatást és lehetséges, hogy az ügyfélnek várnia kell, amíg a kiszolgáló válaszol.
    \item Táskák: A ROS kommunikációs adatok tárolása és lejátszása táskákban történik. A táskák kulcsfontosságú adattárolási mechanizmust jelentenek az érzékelőadatok számára, amelyek megszerzése kihívást jelenthet, de a robotalgoritmusok létrehozásához és teszteléséhez szükségesek. A bonyolult robotmechanizmusokkal való munka során a táskák rendkívül praktikus funkciót jelentenek.
\end{itemize}
Az ilyen típusú gráfok az \href{http://wiki.ros.org/rqt_graph}{rqt\_graph} nevű eszközzel generálhatók.

\subsection{A ROS csomópontok}
A ROS klienskönyvtárak, például a roscpp és a rospy használatával hozhatunk létre ROS-csomópontokat, amelyek számításokat végeznek. A ROS-témákat, szolgáltatásokat és paramétereket egy csomópont használhatja más csomópontokhoz való kapcsolódáshoz. Egy robotban számos csomópont lehet jelen, például olyanok, amelyek feldolgozzák a kameraképeket, kezelik a robot soros kommunikációját, kiszámítják az odometriát stb.

A rendszer hibatűrővé tehető a csomópontok használatával. Egy teljes robotrendszer akkor is tovább működhet, ha egy csomópont összeomlik. Mivel minden egyes csomópont csak egy funkciót kezel, a csomópontok megkönnyítik a hibakeresést.

Minden működő csomópontnak nevet kell adni, hogy a rendszer többi része felismerhesse. Például a /camera\_node lehet egy olyan csomópont neve, amely kameraképeket sugároz.

A rosbash eszközzel a ROS-csomópontokat vizsgálhatjuk. A rosnode parancs segítségével információkat kaphatunk egy ROS-csomópontról. Itt vannak a rosnode használatai
\begin{itemize}
    \item
    \begin{verbatim}
        $ rosnode info [node_name]
    \end{verbatim}ez kiírja a csomópontra vonatkozó információkat
    \item
    \begin{verbatim}
        $ rosnode kill [node_name]
    \end{verbatim}Megől egy futó csomópontot
    \item
    \begin{verbatim}
        $ rosnode list
    \end{verbatim}Ez listázza a futó csomópontokat
    \item
    \begin{verbatim}
        $ rosnode machine [machine_name]
    \end{verbatim}Ez felsorolja az adott gépen futó csomópontokat vagy a gépek listáját.
    \item
    \begin{verbatim}
        $ rosnode ping
    \end{verbatim} Ez ellenőrzi a csomópont csatlakoztathatóságát.
    \item
    \begin{verbatim}
        $ rosnode cleanup
    \end{verbatim}Ez eltörli az elérhetetlen csomópontok regisztrációját.
\end{itemize}



\subsection{ROS üzenetek}
A ROS-csomópontok üzeneteket cserélnek egymással egy témába való üzenetküldéssel. Az üzenetek egy egyszerű adatstruktúra, amely mezőtípusokkal rendelkezik, ahogyan azt korábban már tárgyaltuk. A ROS-üzenet a szabványos primitív adattípusokat és a primitív típusokból álló tömböket támogatja.

A szolgáltatáshívások egy másik módszer, amellyel a csomópontok kommunikálhatnak. A szolgáltatások üzenetek, és az srv fájl tartalmazza az egyes szolgáltatási üzenettípusok definícióit.

A következő technikával hozzáférhetünk az üzenetspecifikációhoz. Például az std\_msgs/String használatával megkaphatjuk az std\_msgs/msg/String.msg fájlt. A string üzenet definíciójához a roscpp kliens használata esetén be kell vennünk az std\_msgs/String.h állományt.

Annak megállapításakor, hogy a kiadó és az előfizető ugyanazokat az üzenetadattípusokat cseréli-e ki, a ROS az MD5 ellenőrző összegeket is összehasonlítja.

A ROS-üzenetekkel kapcsolatos információkhoz a ROS beépített eszközöket tartalmaz, a rosmsg-et. A következő parancsokat használhatjuk: 
\begin{itemize}
    \item
    \begin{verbatim}
        $ rosmsg show [message]
    \end{verbatim}Ez mutatja az üzenet leírását
    \item
    \begin{verbatim}
        $ rosmsg list
    \end{verbatim}Ez felsorolja az összes üzenetet
    \item
    \begin{verbatim}
        $ rosmsg md5 [message]
    \end{verbatim}Ez megjeleníti az üzenet md5összegét
    \item
    \begin{verbatim}
        $ rosmsg package [package_name]
    \end{verbatim}Ez felsorolja a csomagban lévő üzeneteket
    \item
    \begin{verbatim}
        $ rosmsg packages [package_1] [package_2]
    \end{verbatim}Ez felsorolja az üzeneteket tartalmazó csomagokat
\end{itemize}

\subsection{ROS témák}
A buszok olyan ROS-témák, amelyeken keresztül a ROS-csomópontok kommunikálnak egymással. A topikok anonim publikálási és feliratkozási képessége azt jelenti, hogy az üzenetek létrehozása és fogyasztása különálló folyamatok. A ROS-csomópontok csak a téma nevét ellenőrzik, valamint azt, hogy a kiadó és a feliratkozó üzenettípusa megegyezik-e; nem érdekli őket, hogy melyik csomópont publikálja vagy iratkozik fel a témákra.

A topikok csak egyirányú kommunikációt tesznek lehetővé; ha kérés-válasz kommunikációt akarunk kiépíteni, akkor ROS-szolgáltatásokat kell használnunk.

A ROS-csomópontok a TCP/IP-alapú TCPROS transzportot használják a topikokhoz való csatlakozáshoz. A ROS-ban ezt a technikát használják alapértelmezett szállítási mechanizmusként. A kommunikáció másik formája az UDPROS, amely kizárólag a távműködtetésre alkalmas, és alacsony késleltetésű, laza szállítást biztosít.

A ROS téma eszközzel a ROS témákról kaphatunk információkat. Példa:
\begin{itemize}
    \item
    \begin{verbatim}
        $ rostopic bw /topic
    \end{verbatim}Ez a parancs megjeleníti az adott téma által használt sávszélességet.
    \item
    \begin{verbatim}
        $ rostopic echo /topic
    \end{verbatim}Ez a parancs kiírja a megadott téma tartalmát.
    \item
    \begin{verbatim}
        $ rostopic find /message_type
    \end{verbatim}Ez a parancs megkeresi a megadott üzenettípust használó témákat.
    \item
    \begin{verbatim}
        $ rostopic hz /topic
    \end{verbatim}Ez a parancs megjeleníti az adott téma publikálási arányát.
    \item
    \begin{verbatim}
        $ rostopic info /topic
    \end{verbatim}Ez a parancs információkat nyomtat ki egy aktív témáról.
    \item
    \begin{verbatim}
        $ rostopic list
    \end{verbatim}Ez a parancs felsorolja az összes aktív témát a ROS rendszerben.
    \item
    \begin{verbatim}
        $ rostopic pub /topic message_type args
    \end{verbatim}Ezzel a paranccsal közzétehet egy értéket egy üzenettípusú témában.
    \item
    \begin{verbatim}
        $ rostopic type /topic
    \end{verbatim}Ez megjeleníti az adott téma üzenettípusát.
\end{itemize}


\subsection{ROS szolgáltatások}
A ROS-szolgáltatásokat minden olyan esetben használni kell, amikor kérés/válasz típusú kommunikációra van szükség. A ROS-témák egyirányú jellege miatt ez a fajta kommunikáció nem lehetséges. Az elosztott rendszerek az esetek többségében ROS-szolgáltatásokat alkalmaznak.

A ROS-szolgáltatásokat egy üzenetpár határozza meg. Egy srv fájlban meg kell adnunk a kérés adattípusát és a visszatérés adattípusát. Egy csomag belső srv alkönyvtárában találhatók az srv fájlok.

Egy csomópont egyszerre szolgál kliensként és szerverként a ROS-szolgáltatások számára, lehetővé téve, hogy a kliensek szolgáltatásokat kérjenek a szerverektől. Az eredmények akkor kerülnek elküldésre a szolgáltatás kliensének, ha a szerver sikeresen végrehajtja a szolgáltatási eljárást.

A következő megközelítéssel például elérhetjük a ROS szolgáltatásdefiníciót, páldául ha a my\_package/Image elérheti a my\_package/srv/Image.srv állományt.

Van egy MD5 ellenőrző összeg, amely a ROS szolgáltatások csomópontjait is ellenőrzi. Ha az összeg megegyezik, csak a szerver válaszolhat az ügyfélnek.

Két ROS-eszköz áll rendelkezésre a ROS-szolgáltatás megismeréséhez. A különböző szolgáltatástípusok megismeréséhez használja az első eszközt, a rossrv-t, amely megegyezik a rosmsg-gel. A következő parancs, a rosservice, az aktuálisan aktív ROS-szolgáltatások listázására és lekérdezésére szolgál.

Az alábbiakban néhány utasítást olvashat, hogyan használhatja a rosservice eszközt, hogy többet tudjon meg a jelenleg aktív szolgáltatásokról:
\begin{itemize}
    \item
    \begin{verbatim}
        $ rosservice call /service args
    \end{verbatim}Ez az eszköz meghívja a szolgáltatást a megadott argumentumokkal.
    \item
    \begin{verbatim}
        $ rosservice find service_type
    \end{verbatim}Ez a parancs a megadott szolgáltatástípusba tartozó szolgáltatásokat keresi.
    \item
    \begin{verbatim}
        $ rosservice info /services
    \end{verbatim}Ez információt nyomtat ki az adott szolgáltatásról.
    \item
    \begin{verbatim}
        $ rosservice list
    \end{verbatim}Ez a parancs felsorolja a rendszerben futó aktív szolgáltatásokat.
    \item
    \begin{verbatim}
        $ rosservice type /service
    \end{verbatim}Ez a parancs kiírja az adott szolgáltatás típusát.
    \item
    \begin{verbatim}
        $ rosservice uri /service
    \end{verbatim}Ez az eszköz kiírja a szolgáltatás ROSRPC URI-ját.
\end{itemize}

\subsection{ROS táskák}
A ROS-ban a témák és szolgáltatások üzenetadatait bag-fájlokban tárolják. A bag fájlokat a.bag kiterjesztés jelöli.

A \href{http://wiki.ros.org/rosbag/Commandline}{rosbag} parancs egy vagy több témára való feliratkozással és az üzenet adatainak a fogadáskor történő elmentésével hoz létre bag-fájlokat. Ez a fájl újra lejátszhatja ugyanazokat a témákat, amelyekből rögzítették őket, vagy újra leképezheti az aktuális témákat.

Az adatnaplózás a rosbag elsődleges felhasználási területe. A robotadatok offline is megjeleníthetők és feldolgozhatók, valamint naplózhatók.

A rosbag fájlokkal való munkához a rosbag parancsra van szükség. A bag fájl rögzítésére és lejátszására szolgáló parancsok a következők:
\begin{itemize}
    \item \begin{verbatim}
        $ rosbag record [topic_1] [topic_2] -o [bag_name]
    \end{verbatim}A megadott témákat ez a parancs rögzíti a parancs által megadott zsákfájlba. A -a argumentummal bármilyen témát rögzíthetünk.
    \item \begin{verbatim}
        $ rosbag play [bag_name]
    \end{verbatim}Lejátssza a meglévő táska tartalmát.
\end{itemize}
A GUI eszköz amivel táskákkal dolgozhatunk az \href{http://wiki.ros.org/rqt_bag}{rqt\_bag}.

\subsection{ROS mester}
A DNS-kiszolgáló hasonló a ROS Masterhez. Minden induló ROS-csomópont elkezdi megkeresni a ROS Master-t és regisztrálni a nevét. Ennek eredményeképpen a ROS Master minden olyan csomópontról tud, amely jelenleg aktív a ROS-rendszerben. Visszahívást generál, és frissíti a legfrissebb információkkal, amikor bármelyik csomópont adatai megváltoznak. Az egyes csomópontokhoz való csatlakozáshoz ezek a csomópontadatok hasznosak.

Amikor egy csomópont elkezd közzétenni egy témát, a csomópont értesíti a ROS Master-t a téma nevéről és adattípusáról. Ha vannak további csomópontok, amelyek feliratkoztak ugyanarra a témára, a ROS Master ellenőrzi őket. A kiadó csomópont információit a ROS Master megosztja az előfizető csomóponttal, ha ugyanarra a témára több csomópont is előfizetett. A két csomópont a csomópontadatok átvétele után a TCP/IP-alapú TCPROS protokoll segítségével csatlakozik. A ROS Master nem játszik szerepet a két csomópont kezelésében, miután azok összekapcsolódtak. A preferenciáinktól függően megállíthatjuk akár a kiadó csomópontot, akár az előfizető csomópontot. Újra ellenőrzi a ROS Masterrel, ha bármelyik csomópont leállt. A ROS-szolgáltatások ugyanezt az eljárást alkalmazzák.

A ROS klienskönyvtárak, köztük a roscpp és a rospy a csomópontok létrehozására szolgálnak. Ezek a kliensek XMLRPC-alapú API-kon keresztül kommunikálnak a ROS Masterrel, amelyek a ROS rendszer API-k rendszerháttereként szolgálnak.

A ROS Master IP-címe és portja a ROS\_MASTER\_URI környezeti változóban található. Ez a változó lehetővé teszi, hogy a ROS-csomópontok megtalálják a ROS Master-t. A csomópontok közötti kommunikáció nem fog megtörténni, ha ez a változó ki van kapcsolva. A localhost IP-cím vagy név használható, ha a ROS-t egyetlen rendszerben használjuk. Azonban egy elosztott hálózatban, ahol a feldolgozás sok fizikai gépen zajlik, helyesen kell definiálnunk a ROS\_MASTER\_URI-t. Csak így lesznek képesek a távoli csomópontok megtalálni és kommunikálni egymással. Egy elosztott rendszerben csak egy Masterre van szükségünk, és ahhoz, hogy a távoli ROS-csomópontok elérjék a Master-t, annak olyan számítógépen kell futnia, amelyet az összes többi számítógép sikeresen tud pingelni.
\cite{joseph:ros}
\bibliography{src/referenciák}
\end{document}
